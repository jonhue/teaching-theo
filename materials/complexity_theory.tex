\documentclass{beamer}
%
% Choose how your presentation looks.
%
% For more themes, color themes and font themes, see:
% http://deic.uab.es/~iblanes/beamer_gallery/index_by_theme.html
%
\mode<presentation>
{
  \usetheme{default}      % or try Darmstadt, Madrid, Warsaw, ...
  \usecolortheme{default} % or try albatross, beaver, crane, ...
  \usefonttheme{default}  % or try serif, structurebold, ...
  \setbeamertemplate{navigation symbols}{}
  \setbeamertemplate{caption}[numbered]
  \setbeamertemplate{footline}[frame number]
  \setbeamertemplate{itemize items}[circle]
  \setbeamertemplate{theorems}[numbered]
  \setbeamercolor*{structure}{bg=white,fg=blue}
  \setbeamerfont{block title}{size=\normalsize}
  \setbeamertemplate{frametitle continuation}{}
}

\newtheorem{proposition}[theorem]{Proposition}
\theoremstyle{definition}
\newtheorem{algorithm}[theorem]{Algorithm}
\newtheorem{idea}[theorem]{Idea}

\usepackage[english]{babel}
\usepackage[utf8]{inputenc}
\usepackage[T1]{fontenc}
\usepackage{aligned-overset}
\usepackage{alltt}
\usepackage{amsmath}
\usepackage{csquotes}
\usepackage{multicol}
\usepackage{stmaryrd}
\usepackage{tabularx}

\renewcommand\tabularxcolumn[1]{m{#1}}
\newcolumntype{R}{>{\raggedleft\arraybackslash}X}

\def\code#1{\texttt{\frenchspacing#1}}
\def\padding{\vspace{0.5cm}}
\def\spadding{\vspace{0.25cm}}
\def\b{\textcolor{blue}}
\def\r{\textcolor{red}}
\def\g#1{{\usebeamercolor[fg]{block title example}{#1}}}

% fix for \pause in align
\makeatletter
\let\save@measuring@true\measuring@true
\def\measuring@true{%
  \save@measuring@true
  \def\beamer@sortzero##1{\beamer@ifnextcharospec{\beamer@sortzeroread{##1}}{}}%
  \def\beamer@sortzeroread##1<##2>{}%
  \def\beamer@finalnospec{}%
}
\makeatother

% \usepackage[sorting=ynt,style=alphabetic]{biblatex}
% \addbibresource{sources.bib}

\title[Theoretical Computer Science]{Theoretical Computer Science \\ Complexity Theory}
\author{Jonas Hübotter}
\date{}

\begin{document}

\begin{frame}
  \titlepage
\end{frame}

% Uncomment these lines for an automatically generated outline.
\begin{frame}[allowframebreaks]{Outline}
 \tableofcontents[subsubsectionstyle=hide]
\end{frame}
% \AtBeginSection[]
%   {
%      \begin{frame}
%      \frametitle{Plan}
%      \tableofcontents[currentsection, sectionstyle=show/hide, hideothersubsections]
%      \end{frame}
%   }

\section{$\mathcal{P}$}

\begin{frame}{$\mathcal{P}$}
    $\mathcal{P}$ is the class of problems that can be solved by a DTM in polynomial time.\pause
    \begin{definition}
        We define \begin{align*}
            \text{time}_M(w)  = (&\text{\#steps until DTM $M[w]$ halts}) \in \mathbb{N} \cup \{\infty\} \pause\\
            \text{TIME}(f(n)) = \{&A \subseteq \Sigma^* \mid \exists \text{DTM}\ M.\ A = L(M) \land \forall w \in \Sigma^*.\ \\&\text{time}_M(w) \leq f(|w|)\} \\&\text{for a total function $f : \mathbb{N} \to \mathbb{N}$}
        \end{align*}\pause
        Then, the complexity class $\mathcal{P}$ is given as \begin{align*}
            \mathcal{P} = \bigcup_{p \in \text{polynomial}} \text{TIME}(p(n)) \pause= \bigcup_{k \geq 0} \text{TIME}(\mathcal{O}(n^k))
        \end{align*} where $\text{TIME}(\mathcal{O}(f)) = \bigcup_{g \in \mathcal{O}(f)} \text{TIME}(g)$.
    \end{definition}
\end{frame}

\section{$\mathcal{NP}$}

\begin{frame}{$\mathcal{NP}$}
    $\mathcal{NP}$ is the class of problems that can be solved by a NTM in polynomial time.\pause
    \begin{definition}
        We define \begin{align*}
            \text{ntime}_M(w) = \begin{cases}
                (\text{minimal \#steps for NTM $M[w]$ to halt}) & w \in L(M) \\
                0 & w \not\in L(M)
            \end{cases}
        \end{align*}\pause \begin{align*}
            \text{NTIME}(f(n)) = \{&A \subseteq \Sigma^* \mid \exists \text{NTM}\ M.\ A = L(M) \land \forall w \in \Sigma^*.\ \\&\text{ntime}_M(w) \leq f(|w|)\} \\&\text{for a total function $f : \mathbb{N} \to \mathbb{N}$}
        \end{align*}
    \end{definition}
\end{frame}

\begin{frame}{$\mathcal{NP}$}
    To simplify the notation, we do not require that the NTM $M$ terminates for inputs $w \not\in L(M)$. This is not a restriction as we can always define the NTM $M'$ which returns $0$ after $p(|w|)$ steps (timeout).\pause
    \begin{definition}
        The complexity class $\mathcal{NP}$ is given as \begin{align*}
            \mathcal{NP} = \bigcup_{p \in \text{polynomial}} \text{NTIME}(p(n)) \pause= \bigcup_{k \geq 0} \text{NTIME}(\mathcal{O}(n^k)).
        \end{align*}
    \end{definition}
\end{frame}

\subsection{Polynomially Bounded Verifier}

\begin{frame}{Polynomially Bounded Verifier}
    A problem $A$ is in $\mathcal{NP}$ if and only if solutions (which are described by \emph{certificates}) to the problem can be verified in polynomial time by a DTM (a \emph{polynomially bounded verifier}).\pause\par\spadding
    \begin{block}{Intuition}
        \begin{itemize}
            \item A decision problem can be thought of an exploration of the search space consisting of all instances with the goal of finding a solution.\pause
            \item Problems in $\mathcal{NP}$ may be harder than problems in $\mathcal{P}$ as a NTM is able to pursue exponentially many paths in the search tree.\pause
            \item However, once the NTM found a path, the length of this path must be polynomial in the size of the input.\pause
            \item Thus, a DTM must be able to verify that a given path is correct in polynomial time.
        \end{itemize}
    \end{block}
\end{frame}

\begin{frame}{Polynomially Bounded Verifier}
    \begin{definition}
        Let $M$ be a DTM with $L(M) = \{w\#c \mid w \in \Sigma^*, c \in \Delta^*\}$.\pause
        \begin{itemize}
            \item If $w\#c \in L(M)$, $c$ is a \b{certificate} for $w$.\pause
            \item $M$ is a \b{polynomially bounded verifier} for the language $\{w \in \Sigma^* \mid \exists c \in \Delta^*.\ w\#c \in L(M)\}$ (i.e. the language of all words that have a certificate) if there exists a polynomial $p$ such that $\text{time}_M(w\#c) \leq p(|w|)$.
        \end{itemize}\pause
        \r{Especially: $c \leq p(|w|)$, i.e. the size of the certificate must be polynomially bounded by the size of the input.}
    \end{definition}
\end{frame}

\subsection{Polynomial Reduction}

\begin{frame}{Polynomial Reduction}
    \begin{definition}
        Given problems $A \subseteq \Sigma^*, B \subseteq \Gamma^*$, A is \b{polynomially reducable} to $B$ (denoted $A \leq_p B$) if there exists a total and by a DTM in polynomial time computable function $f : \Sigma^* \to \Gamma^*$ such that \begin{align*}
            \forall w \in \Sigma^*.\ w \in A \iff f(w) \in B.
        \end{align*}
    \end{definition}\pause
    The complexity classes $\mathcal{P}$ and $\mathcal{NP}$ are closed under polynomial reduction.
\end{frame}

\subsection{$\mathcal{NP}$-completeness}

\begin{frame}{$\mathcal{NP}$-completeness}
    \begin{definition}
        \begin{itemize}
            \item The language $L$ is \b{$\mathcal{NP}$-hard} if $\forall A \in \mathcal{NP}.\ A \leq_p L$.\pause
            \item The language $L$ is \b{$\mathcal{NP}$-complete} if $L \in \mathcal{NP}$ and $L$ is $\mathcal{NP}$-hard.
        \end{itemize}
    \end{definition}\pause
    \begin{example}[Proving $\mathcal{NP}$-completeness]\pause
        \begin{itemize}
            \item If $A \leq_p B$ and $A$ is $\mathcal{NP}$-hard, then $B$ is $\mathcal{NP}$-hard.\pause
            \item If $A \leq_p B$ and $B \in \mathcal{NP}$, then $A \in \mathcal{NP}$.\pause
            \item If there exists a polynomially bounded verifier for $A$, then $A \in \mathcal{NP}$.
        \end{itemize}
    \end{example}
\end{frame}

\section{Approximation Algorithm}

\begin{frame}{Approximation Algorithm}
    \begin{definition}
        A \b{$d$-approximation algorithm} ($d \in \mathbb{R}$) for an optimization problem is an algorithm that computes in polynomial time a solution to the problem that is at most $d$ times worse than the optimal solution.
    \end{definition}
\end{frame}

% \begin{frame}[allowframebreaks]{References}

% \printbibliography

% \end{frame}

\end{document}
