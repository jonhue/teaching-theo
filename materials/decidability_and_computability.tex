\documentclass{beamer}
%
% Choose how your presentation looks.
%
% For more themes, color themes and font themes, see:
% http://deic.uab.es/~iblanes/beamer_gallery/index_by_theme.html
%
\mode<presentation>
{
  \usetheme{default}      % or try Darmstadt, Madrid, Warsaw, ...
  \usecolortheme{default} % or try albatross, beaver, crane, ...
  \usefonttheme{default}  % or try serif, structurebold, ...
  \setbeamertemplate{navigation symbols}{}
  \setbeamertemplate{caption}[numbered]
  \setbeamertemplate{footline}[frame number]
  \setbeamertemplate{itemize items}[circle]
  \setbeamertemplate{theorems}[numbered]
  \setbeamercolor*{structure}{bg=white,fg=blue}
  \setbeamerfont{block title}{size=\normalsize}
  \setbeamertemplate{frametitle continuation}{}
}

\newtheorem{proposition}[theorem]{Proposition}
\theoremstyle{definition}
\newtheorem{algorithm}[theorem]{Algorithm}
\newtheorem{idea}[theorem]{Idea}

\usepackage[english]{babel}
\usepackage[utf8]{inputenc}
\usepackage[T1]{fontenc}
\usepackage{aligned-overset}
\usepackage{alltt}
\usepackage{amsmath}
\usepackage{csquotes}
\usepackage{multicol}
\usepackage{stmaryrd}
\usepackage{tabularx}

\renewcommand\tabularxcolumn[1]{m{#1}}
\newcolumntype{R}{>{\raggedleft\arraybackslash}X}

\def\code#1{\texttt{\frenchspacing#1}}
\def\padding{\vspace{0.5cm}}
\def\spadding{\vspace{0.25cm}}
\def\b{\textcolor{blue}}
\def\r{\textcolor{red}}
\def\g#1{{\usebeamercolor[fg]{block title example}{#1}}}

% fix for \pause in align
\makeatletter
\let\save@measuring@true\measuring@true
\def\measuring@true{%
  \save@measuring@true
  \def\beamer@sortzero##1{\beamer@ifnextcharospec{\beamer@sortzeroread{##1}}{}}%
  \def\beamer@sortzeroread##1<##2>{}%
  \def\beamer@finalnospec{}%
}
\makeatother

% \usepackage[sorting=ynt,style=alphabetic]{biblatex}
% \addbibresource{sources.bib}

\title[Theoretical Computer Science]{Theoretical Computer Science}
\author{Jonas Hübotter}
\date{}

\begin{document}

\begin{frame}
  \titlepage
\end{frame}

% Uncomment these lines for an automatically generated outline.
\begin{frame}[allowframebreaks]{Outline}
  \tableofcontents[subsubsectionstyle=hide]
 \end{frame}
% \AtBeginSection[]
%   {
%      \begin{frame}
%      \frametitle{Plan}
%      \tableofcontents[currentsection, sectionstyle=show/hide, hideothersubsections]
%      \end{frame}
%   }

\section{Decidability and Computability}

\subsection{Problems}

\begin{frame}{Problems}

Given problem $A \subseteq \Sigma^*$.\spadding

\begin{definition}
$x$ is an \b{instance} of $A$ if $x \in \Sigma^*$.\par
$x$ is a \b{solution} to $A$ if $x \in A$.
\end{definition}

\end{frame}

\subsection{Decidability}

\begin{frame}{Decidability}

\begin{definition}
The \b{characteristic function} of a problem $A$ is given as
\begin{align*}
    \chi_A(x) = \begin{cases}
                    1 & \text{if}\ x \in A \\
                    0 & \text{otherwise}
                \end{cases}.
\end{align*}
\end{definition}

\begin{definition}
A problem $A$ is \b{decidable} if its characteristic function is computable.
\end{definition}

\end{frame}

\section{Problems}

\subsection{Undecidable problems}

\begin{frame}[allowframebreaks]{Undecidable problems}

\begin{block}{Special Halting Problem}
$K = \{w \in \{0,1\}^* \mid M_w[w]\downarrow\}$
\end{block}

\begin{block}{General Halting Problem}
$H = \{w\#x \mid w,x \in \{0,1\}^* \land M_w[x]\downarrow\}$
\end{block}

\begin{block}{Halting Problem on an empty tape}
$H_0 = \{w \in \{0,1\}^* \mid M_w[\epsilon]\downarrow\}$
\end{block}

\begin{block}{Post's Correspondence Problem (PCP)}
given: finite sequence $(x_1, y_1), \ldots, (x_k, y_k)$ where $x_i,y_i \in \Sigma^+$\par
problem: is there a sequence of indices $i_1,\ldots,i_n \in [k], n > 0$ such that $x_{i_1} \dots x_{i_n} = y_{i_1} \dots y_{i_n}$?
\end{block}

\framebreak

\begin{block}{CFG problems}
Let $G, G_1, G_2$ be CFGs.
\begin{itemize}
    \item $L(G_1) \cap L(G_2) = \emptyset$?
    \item $|L(G_1) \cap L(G_2)| = \infty$?
    \item $L(G_1) \cap L(G_2)$ context-free?
    \item $L(G_1) \subseteq L(G_2)$?
    \item $L(G_1) = L(G_2)$?
    \item $G$ ambiguous?
    \item $G$ regular?
    \item $G$ deterministic?
    \item for some regular expression $\alpha$, $L(G) = L(\alpha)$?
\end{itemize}
\end{block}

\end{frame}

\subsection{NP-complete problems}

\begin{frame}[allowframebreaks]{NP-complete problems}

\begin{block}{SAT}
given: propositional formula $F$\par
problem: is $F$ satisfiable?+-
\end{block}

\begin{block}{CNF-SAT}
given: propositional formula $F$ in $k$CNF for $k \geq 3$\par
problem: is $F$ satisfiable?
\end{block}

\begin{block}{NONEQUIVALENCE}
given: two propositional formulas $F_1, F_2$\par
problem: is there an assignment $\mathcal{A}$ such that $\mathcal{A}(F_1) \neq \mathcal{A}(F_2)$?
\end{block}

\framebreak

\begin{block}{HAMILTON}
given: undirected graph $G$\par
problem: does $G$ have a Hamiltonian circuit (i.e. a circuit visiting every vertex exactly once)?
\end{block}

\begin{block}{TRAVELLING SALESMAN (TSP)}
given: matrix $(M_{ij})_{1\leq i,j \leq n}$ of \textit{distances}, $k \in \mathbb{N}$\par
problem: is there a roundtrip (Hamiltonian circuit) of length $\leq k$?
\end{block}

\framebreak

\begin{block}{COL}
given: undirected graph $G$, $k \geq 3$\par
problem: is there a vertex coloring with $k$ colors such that no two adjacent vertices are assigned the same color?
\end{block}

\begin{block}{SETCOVER}
given: $T_1,\ldots,T_n \subseteq M$ with $M$ finite, $k \in \mathbb{N}$\par
problem: is there $i_1,\ldots,i_n \in [k]$ with $M = T_{i_1} \cup \dots \cup T_{i_n}$?
\end{block}

\begin{block}{CLIQUE}
given: undirected graph $G$, $k \in \mathbb{N}$\par
problem: does $G$ have a clique of at least size $k$?
\end{block}

\framebreak

\begin{block}{KNAPSACK}
given: $a_1,\ldots,a_n,b \in \mathbb{N}$\par
problem: is there $R \subseteq [n]$ with $\sum_{i \in R} a_i = b$?
\end{block}

\begin{block}{PARTITION}
given: $a_1,\ldots,a_n \in \mathbb{N}$\par
problem: is there $I \subseteq [n]$ with $\sum_{i \in I} a_i = \sum_{i \not\in I} a_i$?
\end{block}

\begin{block}{BINPACKING}
given: \textit{can size} $b \in \mathbb{N}$, \textit{\# of cans} $k \in \mathbb{N}$, \textit{objects} $a_1,\ldots,a_n \in \mathbb{N}$\par
problem: can each object be assigned to a can without any can overflowing?
\end{block}

\end{frame}

% \begin{frame}[allowframebreaks]{References}

% \printbibliography

% \end{frame}

\end{document}
