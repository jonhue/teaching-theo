\documentclass{beamer}
%
% Choose how your presentation looks.
%
% For more themes, color themes and font themes, see:
% http://deic.uab.es/~iblanes/beamer_gallery/index_by_theme.html
%
\mode<presentation>
{
  \usetheme{default}      % or try Darmstadt, Madrid, Warsaw, ...
  \usecolortheme{default} % or try albatross, beaver, crane, ...
  \usefonttheme{default}  % or try serif, structurebold, ...
  \setbeamertemplate{navigation symbols}{}
  \setbeamertemplate{caption}[numbered]
  \setbeamertemplate{footline}[frame number]
  \setbeamertemplate{itemize items}[circle]
  \setbeamertemplate{theorems}[numbered]
  \setbeamercolor*{structure}{bg=white,fg=blue}
  \setbeamerfont{block title}{size=\normalsize}
  \setbeamertemplate{frametitle continuation}{}
}

\newtheorem{proposition}[theorem]{Proposition}
\theoremstyle{definition}
\newtheorem{algorithm}[theorem]{Algorithm}
\newtheorem{idea}[theorem]{Idea}

\usepackage[english]{babel}
\usepackage[utf8]{inputenc}
\usepackage[T1]{fontenc}
\usepackage{aligned-overset}
\usepackage{alltt}
\usepackage{amsmath}
\usepackage{csquotes}
\usepackage{multicol}
\usepackage{stmaryrd}
\usepackage{tabularx}

\renewcommand\tabularxcolumn[1]{m{#1}}
\newcolumntype{R}{>{\raggedleft\arraybackslash}X}

\def\code#1{\texttt{\frenchspacing#1}}
\def\padding{\vspace{0.5cm}}
\def\spadding{\vspace{0.25cm}}
\def\b{\textcolor{blue}}
\def\r{\textcolor{red}}
\def\g#1{{\usebeamercolor[fg]{block title example}{#1}}}

% fix for \pause in align
\makeatletter
\let\save@measuring@true\measuring@true
\def\measuring@true{%
  \save@measuring@true
  \def\beamer@sortzero##1{\beamer@ifnextcharospec{\beamer@sortzeroread{##1}}{}}%
  \def\beamer@sortzeroread##1<##2>{}%
  \def\beamer@finalnospec{}%
}
\makeatother

% \usepackage[sorting=ynt,style=alphabetic]{biblatex}
% \addbibresource{sources.bib}

\title[Theoretical Computer Science]{Theoretical Computer Science \\ Decidability and Computability}
\author{Jonas Hübotter}
\date{}

\begin{document}

\begin{frame}
  \titlepage
\end{frame}

% Uncomment these lines for an automatically generated outline.
\begin{frame}[allowframebreaks]{Outline}
 \tableofcontents[subsubsectionstyle=hide]
\end{frame}
% \AtBeginSection[]
%   {
%      \begin{frame}
%      \frametitle{Plan}
%      \tableofcontents[currentsection, sectionstyle=show/hide, hideothersubsections]
%      \end{frame}
%   }

\section{Turing Machine (TM)}

\begin{frame}{TM}
    \begin{definition}
        A \b{Turing machine (TM)} $M = (Q, \Sigma, \Gamma, \delta, q_0, \square, F)$\pause\ consists of
        \begin{itemize}
            \item a finite set of \b{states} $Q$\pause;
            \item a (finite) \b{input alphabet} $\Sigma$\pause;
            \item a (finite) \b{tape alphabet} $\Gamma$\pause;
            \item a (partial) \b{transition function} $\delta: Q \times \Gamma \to Q \times \Gamma \times \{L,R,N\}$\pause;
            \item an \b{initial state} $q_0 \in Q$\pause;
            \item an \b{empty tape element} $\square \in \Gamma \setminus \Sigma$\pause; and
            \item a set of \b{terminal (accepting) states} $F \subseteq Q$.
        \end{itemize}
    \end{definition}\pause
    We assume $\delta(q, a) = \bot$ (is undefined) for any $q \in F$, i.e. as soon as we reach a final state the TM halts.\pause\par
    Graphically, transitions are denoted as $\alpha/\beta, \xi$ where $\alpha \in \Gamma$ is the current tape element which is replaced by $\beta \in \Gamma$ and the head moves in the direction $\xi \in \{L,R,N\}$.
\end{frame}

\begin{frame}{TM}
    A Turing machine can be interpreted as a read-write-head operating on an infinite tape initialized with $\square$. $L, R$, and $N$ denote the movement of the head on the tape in the direction left, right, and none, respectively.\pause\par\spadding
    \begin{definition}
        A \b{nondeterministic} TM has the transition function $\delta : Q \times \Gamma \to 2^{Q \times \Gamma \times \{L,R,N\}}$, similarly to nondeterministic PDAs.
    \end{definition}\pause
    \begin{theorem}
        For every nondeterministic TM $N$ there exists a deterministic TM $M$ such that $L(N) = L(M)$.
    \end{theorem}\pause
    Idea: $M$ uses breadth-first search to emulate $N$ (see \textit{dovetailing}).
\end{frame}

\begin{frame}{TM}
    \begin{definition}
        The \b{configuration} of a TM $M$ is a triple $(\alpha, q, \beta)$ where $q \in Q$ is its state, $\alpha \in \Gamma^*$ is the tape content left-to-right up to the position of the head, and $\beta \in \Gamma^*$ is the tape content left-to-right from the element at the position of the head.\pause\par
        Given configuration $(\alpha,q\beta)$, $M$ can be graphically represented as $\cdots \square \alpha \beta \square \cdots$ where $M$ is in state $q$ and its head is at the leftmost symbol of $\beta$.\pause\par\spadding
        The \b{initial configuration} of $M$ on input $w \in \Sigma^*$ is $(\epsilon,q_0,w)$.\pause
        The run of a Turing machine on input $w$ is denoted by $M[w]$.
    \end{definition}\pause
    \begin{definition}
        A TM \b{terminates} when it reaches a configuration $(\alpha,q,a \beta)$ where $\delta(q,a) = \bot$ or $\delta(q,a) = \emptyset$.\pause\ This is denoted by $M[w]\joinrel\downarrow$.
    \end{definition}
\end{frame}

\begin{frame}{TM}
    \begin{definition}
        A \b{run} of a TM $M$ is modeled as the relation $\to_M$. Given $\delta(q, \text{first}(\beta)) = (q',c,D)$ \begin{align*}
            \alpha,q\beta) \to_M \begin{cases}
                (\alpha,q',c\ \text{rest}(\beta)) & D = N \\
                (\alpha c,q',\text{rest}(\beta)) & D = R \\
                (\text{butlast}(\alpha),q',\text{last}(\alpha)\ c\ \text{rest}(\beta)) & D = L
            \end{cases}
        \end{align*} where for $w = w_1 \cdots w_n$, $\text{first}(w) = w_1$, $\text{rest}(w) = w_2 \cdots w_n$, $\text{last}(w) = w_n$, and $\text{butlast}(w) = w_1 \cdots w_{n-1}$.
    \end{definition}
\end{frame}

\begin{frame}{TM}
    \begin{definition}
        A TM $M$ \b{accepts} the language \begin{align*}
            L(M) = \{w \in \Sigma^* \mid \exists q \in F.\ \alpha, \beta \in \Gamma^*.\ (\epsilon,q_0,w) \to_M^* (\alpha,q,\beta)\}.
        \end{align*}
    \end{definition}\pause
    The languages accepted by a TM are precisely the type-0 grammars in the Chomsky-Hierarchy (i.e. semi-decidable languages).
\end{frame}

\subsection{Encoding}

\begin{frame}{Encoding}
    A TM can be encoded using words over the alphabet $\{0,1\}$.\pause
    \begin{definition}
        $M_w$ denotes the Turing machine represented by the encoding $w \in \{0,1\}^*$.
    \end{definition}
\end{frame}

\subsection{$k$-tape TM}

\begin{frame}{$k$-tape TM}
    \begin{definition}
        A \b{$k$-tape TM} is a TM that operates on $k$ tapes simultaneously.
    \end{definition}\pause
    \begin{theorem}
        Every $k$-tape TM can be simulated by a $1$-tape TM.
    \end{theorem}
\end{frame}

\section{Computability}

\begin{frame}{Turing-Computability}
    \begin{definition}
        A function $f : \Sigma^* \to \Sigma^*$ ($\Sigma$ is a finite set) is \b{Turing-computable} if there exists a TM $M$ such that $\forall u,v \in \Sigma^*$ \begin{align*}
            f(u) = v \iff \exists q \in F.\ (\epsilon,q_0,u) \to_M^* (\epsilon,q,v).
        \end{align*}\pause
        In particular, any TM computes a function. $\varphi_w$ denotes the function computed by $M_w$.
    \end{definition}\pause\spadding
    Thus, Turing-computability is a property of functions operating on discrete sets (i.e. functions implemented by a computer).\pause\par\spadding
    The \b{Church-Turing (hypo-)thesis} states that any such function can be computed by a \textit{computer} (or effective method) iff it is Turing-computable (i.e. can be computed by a Turing machine).
\end{frame}

\section{Decidability}

\subsection{Problem}

\begin{frame}{Problem}
\begin{definition}
A \b{problem} is a language $A = \{x \in \Sigma^* \mid P(x)\} \subseteq \Sigma^*$ for some predicate $P : \Sigma^* \to \{0,1\}$.\pause\spadding

Given problem $A \subseteq \Sigma^*$.\pause
\begin{itemize}
    \item $x$ is an \b{instance} of $A$ if $x \in \Sigma^*$.\pause
    \item $x$ is a \b{solution} to $A$ if $x \in A$.
\end{itemize}
\end{definition}
\end{frame}

\subsection{Reduction}

\begin{frame}{Reduction}
    \begin{definition}
        A problem $A \subseteq \Sigma^*$ is \b{reducable} to a problem $B \subseteq \Gamma^*$ (denoted $A \leq B$) if there is a total and computable function $f : \Sigma^* \to \Gamma^*$ such that \begin{align*}
            \forall w \in \Sigma^*.\ w \in A \iff f(w) \in B.
        \end{align*}
    \end{definition}\pause
    \begin{example}
        To show that a function $f$ is a valid reduction from $A$ to $B$ we need to prove three properties:\pause
        \begin{itemize}
            \item $f$ is \textit{total} on $\Sigma^*$\pause;
            \item $f$ is \textit{computable}\pause; and
            \item $f$ is \textit{correct}, i.e. $\forall w \in \Sigma^*.\ w \in A \iff f(w) \in B$.
        \end{itemize}
    \end{example}
\end{frame}

\subsection{Decidability}

\begin{frame}{Decidability}
    Decidability can be interpreted as computability in the context of problems instead of functions.\pause
    \begin{definition}
        The \b{characteristic function} of a problem $A$ is given as
        \begin{align*}
            \chi_A(x) = \begin{cases}
                            1 & x \in A \\
                            0 & x \not\in A
                        \end{cases}.
        \end{align*}
    \end{definition}\pause
    \begin{definition}
        A problem $A$ is \b{decidable} if its characteristic function is computable.
    \end{definition}\pause\spadding
    Given a reduction $A \leq B$,\pause
    \begin{itemize}
        \item $B$ decidable $\implies$ $A$ decidable\pause; and
        \item $A$ undecidable $\implies$ $B$ undecidable.
    \end{itemize}
\end{frame}

\subsection{Theorem of Rice}

\begin{frame}{Theorem of Rice}
    \begin{theorem}[Theorem of Rice]
        Let $\mathcal{F}$ be a set of computable functions.\pause\par
        If $\mathcal{F}$ is non-trivial, i.e. $\mathcal{F} \neq \emptyset$ and $\mathcal{F} \neq \{f \mid f \text{computable}\}$,\pause\par
        then deciding if $\varphi_w \in \mathcal{F}$ is undecidable.\pause\par\spadding
        In other words, \begin{align*}
            C_{\mathcal{F}} = \{w \in \{0,1\}^* \mid \varphi_w \in \mathcal{F}\}
        \end{align*} is undecidable.
    \end{theorem}
\end{frame}

\begin{frame}{Theorem of Rice}
    \begin{example}
        When using the theorem of Rice to prove that a problem $A = \{w \in \{0,1\}^* \mid P(w)\}$ is undecidable, we must complete two steps:\pause
        \begin{enumerate}
            \item construct the set of computable functions $\mathcal{F}$ that fulfill the same property $P$ as functions $\varphi_w$ whose $w$ are in $A$\pause; and
            \item show that $\mathcal{F}$ is non-trivial by giving an example of a computable function $g \in \mathcal{F}$ and a computable function $h \not\in \mathcal{F}$.
        \end{enumerate}\pause
        Note that for step 1, $P$ must \r{not} depend directly on the encoding $w$ but only on $\varphi_w$, otherwise the theorem of Rice cannot be applied.
    \end{example}
\end{frame}

\subsection{Semi-Decidability}

\begin{frame}{Semi-Decidability}
    \begin{definition}
        A problem $A$ is \b{semi-decidable} if \begin{align*}
            \chi'_A(x) = \begin{cases}
                            1 & x \in A \\
                            \bot & x \not\in A
                        \end{cases}.
        \end{align*} is computable.
    \end{definition}\pause\spadding
    \begin{itemize}
        \item Given a reduction $A \leq B$, $B$ semi-decidable $\implies$ $A$ semi-decidable\pause; and
        \item $A$ decidable $\iff$ $A$ semi-decidable and $\bar{A}$ semi-decidable.
    \end{itemize}
\end{frame}

\begin{frame}{Recursive Enumerability}
    \begin{definition}
        A problem $A$ is \b{recursively enumerable} if $A = \emptyset$ or there exists a computable function $f : \mathbb{N}_0 \to A$ such that $A = \{f(0), f(1), \dots\}$.
    \end{definition}\pause
    \begin{theorem}
        A problem $A$ is semi-decidable iff $A$ is recursively enumerable.
    \end{theorem}
\end{frame}

\subsection{Theorem of Rice-Shapiro}

\begin{frame}{Theorem of Rice-Shapiro}
    \begin{theorem}[Theorem of Rice-Shapiro]
        Let $\mathcal{F}$ be a set of computable functions.\pause\par
        If $C_{\mathcal{F}} = \{w \in \{0,1\}^* \mid \varphi_w \in \mathcal{F}\}$ is semi-decidable,\pause\par
        then $f \in \mathcal{F}$ iff there exists a finite and partial function $g \subseteq f$ with $f \in \mathcal{F}$.
    \end{theorem}\pause\spadding
    Often the contrapositive statement is useful:\pause\par
    If there exists an $f \in \mathcal{F}$ such there exists no finite and partial function $g \subseteq f$ with $g \in \mathcal{F}$\pause, then $C_{\mathcal{F}}$ is not semi-decidable.
\end{frame}

\section{Computation Models}

\begin{frame}{Computation Models}
    We have mainly focused on Turing machines to model computability. There are, however, other models for computability that are commonly used:\pause
    \begin{itemize}
        \item \b{WHILE}, programs using \textbf{while} $x \neq 0$ \textbf{do} $\cdots$ \textbf{end while} and \textbf{if} $x = 0$ \textbf{then} $\cdots$ \textbf{else} $\cdots$ \textbf{end if} for control flow\pause;
        \item \b{GOTO}, programs running with a program counter using conditionals (\textbf{if}), commands to jump to a specific line (\textbf{goto}), and commands to terminate (\textbf{halt}) for control flow\pause;
        \item \b{LOOP}, programs using conditionals (\textbf{if}) and loops of a pre-determined fixed length (\textit{loop}) for control flow;
    \end{itemize}
\end{frame}

\begin{frame}{Computation Models}
    \begin{itemize}
        \item \b{primitively recursive (PR)}, functions of the shape \begin{align*}
            f(0,\bar{x}) &= t_0 \\
            f(m+1,\bar{x}) &= t
        \end{align*} where $t_0$ is a term that is only using $x_i$ and other PR functions and $t$ is a term that may use $f(m,\bar{x})$, $x_i$, and other PR functions\pause; and
        \item \b{$\mu$-recursive ($\mu$R)}, an extension of PR where programs are allowed to use the $\mu$-operator which is defined as \begin{align*}
            \mu f(\bar{x} &= \text{find}(0,\bar{x}) \\
            \text{find}(n,\bar{x}) &= \begin{cases}
                n & f(n,\bar{x}) = 0 \\
                \text{find}(n+1,\bar{x}) & \text{otherwise}.
            \end{cases}
        \end{align*}
    \end{itemize}
\end{frame}

\begin{frame}{Computation Models}
    Turing-computable functions are functionally equivalent to \begin{itemize}
        \item Turing machines;
        \item WHILE programs;
        \item GOTO programs; and
        \item $\mu$-recursive programs.
    \end{itemize}\pause\spadding

    LOOP and PR programs are also able to express the same set of functions, but this set is a true subset of all Turing-computable functions.\pause\par
    In other words, there exist Turing-computable functions that are not primitively recursive (or computable by a LOOP program), for example the Ackermann function which is discussed next.
\end{frame}

\section{Ackermann Function}

\begin{frame}{Ackermann Function}
    The Ackermann function can be used to show that a function $f$ is not primitively recursive.\pause
    \begin{definition}
        The \b{Ackermann function} $a$ is not primitively recursive and is defined as \begin{align*}
            a(0,n) &= n+1 \\
            a(m+1,0) &= a(m,1) \\
            a(m+1,n+1) &= a(m,a(m+1,n))
        \end{align*}
    \end{definition}\pause
    \begin{theorem}
        For every primitively recursive function $f : \mathbb{N}^k \to \mathbb{N}$ there exists a $t \in \mathbb{N}$ such that $\forall \bar{x} \in \mathbb{N}^k.\ f(\bar{x}) < a(t, \max \bar{x})$.
    \end{theorem}
\end{frame}

% \begin{frame}[allowframebreaks]{References}

% \printbibliography

% \end{frame}

\end{document}
