\documentclass{beamer}
%
% Choose how your presentation looks.
%
% For more themes, color themes and font themes, see:
% http://deic.uab.es/~iblanes/beamer_gallery/index_by_theme.html
%
\mode<presentation>
{
  \usetheme{default}      % or try Darmstadt, Madrid, Warsaw, ...
  \usecolortheme{default} % or try albatross, beaver, crane, ...
  \usefonttheme{default}  % or try serif, structurebold, ...
  \setbeamertemplate{navigation symbols}{}
  \setbeamertemplate{caption}[numbered]
  \setbeamertemplate{footline}[frame number]
  \setbeamertemplate{itemize items}[circle]
  \setbeamertemplate{theorems}[numbered]
  \setbeamercolor*{structure}{bg=white,fg=blue}
  \setbeamerfont{block title}{size=\normalsize}
}

% \newtheorem{proposition}[theorem]{Proposition}
% \theoremstyle{definition}
% \newtheorem{algorithm}[theorem]{Algorithm}
% \newtheorem{idea}[theorem]{Idea}

\usepackage[english]{babel}
\usepackage[utf8]{inputenc}
\usepackage[T1]{fontenc}
\usepackage{aligned-overset}
\usepackage{alltt}
\usepackage{amsmath}
\usepackage{csquotes}
% \usepackage{multicol}
% \usepackage{stmaryrd}
\usepackage{tabularx}

% \renewcommand\tabularxcolumn[1]{m{#1}}
% \newcolumntype{R}{>{\raggedleft\arraybackslash}X}=

\newcommand\logeq{\mathrel{\vcentcolon\Leftrightarrow}}
\def\code#1{\texttt{\frenchspacing#1}}
\def\padding{\vspace{0.5cm}}
\def\spadding{\vspace{0.25cm}}
\def\b{\textcolor{blue}}
\def\r{\textcolor{red}}
\def\g#1{{\usebeamercolor[fg]{block title example}{#1}}}

% fix for \pause in align
\makeatletter
\let\save@measuring@true\measuring@true
\def\measuring@true{%
  \save@measuring@true
  \def\beamer@sortzero##1{\beamer@ifnextcharospec{\beamer@sortzeroread{##1}}{}}%
  \def\beamer@sortzeroread##1<##2>{}%
  \def\beamer@finalnospec{}%
}
\makeatother

\title[Theoretical Computer Science]{Theoretical Computer Science \\ Regular Languages}
\author{Jonas Hübotter}
\date{}

\begin{document}

\begin{frame}
  \titlepage
\end{frame}

\begin{frame}{Outline}
 \tableofcontents[subsectionstyle=hide, subsubsectionstyle=hide]
\end{frame}
% \AtBeginSection[]
%   {
%      \begin{frame}[allowframebreaks]{Plan}
%      \tableofcontents[currentsection, sectionstyle=show/hide, hideothersubsections]
%      \end{frame}
%   }

\section{Overview}

\begin{frame}{Overview}
    \begin{block}{Representations of regular languages}\pause
        \begin{itemize}
            \item Right-Linear Grammars\pause
            \item Deterministic Finite Automaton (DFA)\pause
            \item Nondeterministic Finite Automaton (NFA)\pause
            \item $\epsilon$-NFA\pause
            \item Regular Expression
        \end{itemize}
    \end{block}
\end{frame}

\section{Deterministic Finite Automaton (DFA)}

\begin{frame}{DFA}
    \begin{definition}
        A \b{deterministic finite automaton (DFA)} $M = (Q, \Sigma, \delta, q_0, F)$\pause consists of
        \begin{itemize}
            \item a finite set of \b{states} $Q$\pause;
            \item a (finite) \b{alphabet} $\Sigma$\pause;
            \item a total \b{transition function} $\delta: Q \times \Sigma \to Q$\pause;
            \item an \b{initial state} $q_0 \in Q$\pause; and
            \item a set of \b{terminal (accepting) states} $F \subseteq Q$.
        \end{itemize}
    \end{definition}
\end{frame}

\begin{frame}{DFA}
    \begin{definition}
        The \b{induced transition function} $\hat{\delta}$ of a DFA $M$ is defined by\pause
        \begin{align*}
            \hat{\delta}(q, \epsilon) &= q\pause\\
            \hat{\delta}(q, aw) &= \hat{\delta}(\delta(q,a),w), a \in \Sigma, w \in \Sigma^*.
        \end{align*}\pause
        The language \b{accepted} by $M$ is $L(M) = \{w \in \Sigma^* \mid \hat{\delta}(q_0, w) \in F\}$.
    \end{definition}
\end{frame}

\section{Nondeterministic Finite Automaton (NFA)}

\begin{frame}{NFA}
    \begin{definition}
        A \b{nondeterministic finite automaton (NFA)} $N = (Q, \Sigma, \delta, q_0, F)$\pause consists of
        \begin{itemize}
            \item $Q, \Sigma, q_0, F$ as defined for DFAs\pause; and
            \item a (partial) \b{transition function} $\delta: Q \times \Sigma \to 2^Q$.
        \end{itemize}
    \end{definition}
\end{frame}

\begin{frame}{NFA}
    \begin{definition}
        The \b{induced transition function} $\hat{\bar{\delta}}$ of a NFA $N$ is defined analogously to $\hat{\delta}$ where\pause
        \begin{align*}
            \bar{\delta}: 2^Q \times \Sigma \to 2^Q, (S, a) \mapsto \bigcup_{q \in S} \delta(q, a).
        \end{align*}\pause
        The language \b{accepted} by $N$ is $L(N) = \{w \in \Sigma^* \mid \hat{\bar{\delta}}(\{q_0\}, w) \cap F \neq \emptyset\}$.
    \end{definition}
\end{frame}

\end{document}
