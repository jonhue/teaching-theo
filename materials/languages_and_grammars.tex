\documentclass{beamer}
%
% Choose how your presentation looks.
%
% For more themes, color themes and font themes, see:
% http://deic.uab.es/~iblanes/beamer_gallery/index_by_theme.html
%
\mode<presentation>
{
  \usetheme{default}      % or try Darmstadt, Madrid, Warsaw, ...
  \usecolortheme{default} % or try albatross, beaver, crane, ...
  \usefonttheme{default}  % or try serif, structurebold, ...
  \setbeamertemplate{navigation symbols}{}
  \setbeamertemplate{caption}[numbered]
  \setbeamertemplate{footline}[frame number]
  \setbeamertemplate{itemize items}[circle]
  \setbeamertemplate{theorems}[numbered]
  \setbeamercolor*{structure}{bg=white,fg=blue}
  \setbeamerfont{block title}{size=\normalsize}
}

% \newtheorem{proposition}[theorem]{Proposition}
% \theoremstyle{definition}
% \newtheorem{algorithm}[theorem]{Algorithm}
% \newtheorem{idea}[theorem]{Idea}

\usepackage[english]{babel}
\usepackage[utf8]{inputenc}
\usepackage[T1]{fontenc}
\usepackage{aligned-overset}
\usepackage{alltt}
\usepackage{amsmath}
\usepackage{csquotes}
% \usepackage{multicol}
% \usepackage{stmaryrd}
\usepackage{tabularx}

% \renewcommand\tabularxcolumn[1]{m{#1}}
% \newcolumntype{R}{>{\raggedleft\arraybackslash}X}=

\newcommand\logeq{\mathrel{\vcentcolon\Leftrightarrow}}
\def\code#1{\texttt{\frenchspacing#1}}
\def\padding{\vspace{0.5cm}}
\def\spadding{\vspace{0.25cm}}
\def\b{\textcolor{blue}}
\def\r{\textcolor{red}}
\def\g#1{{\usebeamercolor[fg]{block title example}{#1}}}

% fix for \pause in align
\makeatletter
\let\save@measuring@true\measuring@true
\def\measuring@true{%
  \save@measuring@true
  \def\beamer@sortzero##1{\beamer@ifnextcharospec{\beamer@sortzeroread{##1}}{}}%
  \def\beamer@sortzeroread##1<##2>{}%
  \def\beamer@finalnospec{}%
}
\makeatother

\title[Theoretical Computer Science]{Theoretical Computer Science \\ Languages and Grammars}
\author{Jonas Hübotter}
\date{}

\begin{document}

\begin{frame}
  \titlepage
\end{frame}

\begin{frame}{Outline}
 \tableofcontents[subsectionstyle=hide, subsubsectionstyle=hide]
\end{frame}
% \AtBeginSection[]
%   {
%      \begin{frame}[allowframebreaks]{Plan}
%      \tableofcontents[currentsection, sectionstyle=show/hide, hideothersubsections]
%      \end{frame}
%   }

\section{Formal Languages}

\begin{frame}{Formal Languages}
    \begin{definition}
        \begin{itemize}
            \item The \b{alphabet} $\Sigma$ is a finite set of symbols.\pause
            \item A finite sequence of symbols is called a \b{word}.\pause
            \item $\epsilon$ is the \b{empty word}.\pause
            \item $\Sigma^*$ is the set of all words over an alphabet $\Sigma$.\pause
            \item $L \subseteq \Sigma^*$ is called a \b{(formal) language}.
        \end{itemize}.
    \end{definition}\pause
    \begin{definition}[Operations on words]
        \begin{itemize}
            \item $|w|$ is the \b{length} of word $w$.\pause
            \item $uv$ is the \b{concatenation} of two words $u, v$.\pause
            \item $w^0 = \epsilon, w^{n+1} = ww^n$ defines \b{repetition} of a word $w$.
        \end{itemize}
    \end{definition}
\end{frame}

\begin{frame}{Operations on Formal Languages}
    \begin{definition}
        Let $A$ and $B$ be formal languages.
        \begin{itemize}
            \item $AB = \{uv \mid u \in A, v \in B\}$ (\b{concatenation}).\pause
            \item $A^0 = \{\epsilon\}, A^{n+1} = AA^n$ (\b{repetition}).\pause
            \item $A^* = \bigcup_{n \in \mathbb{N}_0} A^n$.\pause
            \item $A^+ = AA^* = \bigcup_{n \in \mathbb{N}} A^n$.
        \end{itemize}.
    \end{definition}
\end{frame}

\section{Grammars}

\begin{frame}{Grammars}
    \begin{definition}
        A \b{grammar} is a 4-tuple $G = (V, \Sigma, P, S)$\pause where
        \begin{itemize}
            \item $V$ is a finite set of \b{non-terminals} (or \b{variables})\pause;
            \item $\Sigma$ is an alphabet whose symbols are called \b{terminals}\pause;
            \item $P \subseteq (V \cup \Sigma)^* \times (V \cup \Sigma)^*$ is a set of \b{productions}\pause; and
            \item $S \in V$ is the \b{initial variable}.
        \end{itemize}
    \end{definition}
\end{frame}

\begin{frame}{Grammars}
    \begin{definition}[Derivation]
        A grammar induces a \b{derivation relation} $\rightarrow_G$ on words over $V \cup \Sigma$:\pause
        \begin{align*}
            &\forall \alpha, \alpha' \in (V \cup \Sigma)^*.\ \alpha \rightarrow_G \alpha' \logeq \exists \beta \to \beta' \in P.\\
            &\alpha = \alpha_1\beta\alpha_2 \text{ and } \alpha' = \alpha_1\beta'\alpha_2.
        \end{align*}\pause
        A \b{derivation} of $\alpha_n$ from $\alpha_1$ is denoted by $\alpha_1 \rightarrow_G \dots \rightarrow_G \alpha_n$.
    \end{definition}\pause
    \begin{definition}[Language of a Grammar]
        Given the derivation of $\alpha_n$ from $\alpha_1$, $G$ \b{produces} $\alpha_n$ if $\alpha_1 = S$ and $\alpha_n \in \Sigma^*$.\pause\par
        The \b{language} of $G$ $L(G)$ is the set of all words produced by $G$.
    \end{definition}
\end{frame}

\section{Chomsky-Hierarchy}

\begin{frame}{Chomsky-Hierarchy}
    \begin{definition}
        A grammar $G$ is of\pause
        \begin{itemize}
            \item \b{Type 0} always;\pause
            \item \b{Type 1 (context-sensitive)} $\forall \alpha \to \beta \in P\setminus\{S \to \epsilon\}.\ |\alpha| \leq |\beta|$;\pause
            \item \b{Type 2 (context-free)} if of type 1 and $\forall \alpha \to \beta \in P.\ \alpha \in V$;\pause
            \item \b{Type 3 (right-linear)} if of type 2 and $\forall \alpha \to \beta \in P\setminus\{S \to \epsilon\}.\ \beta \in \Sigma \cup \Sigma V$.
        \end{itemize}
    \end{definition}
\end{frame}

\section{Word problem}

\begin{frame}{Word problem}
    \begin{definition}[Word problem]
        given: a grammar $G$, a word $w \in \Sigma^*$\par
        problem: $w \in L(G)$?
    \end{definition}\pause\padding
    Automata are used to solve the word problem. Depending on the type of the grammar different automata are used.\par
    For example, (Non-)Deterministic Finite Automata are used for right-linear grammars (regular languages) while Pushdown Automata are used for context-free grammars.
\end{frame}

\end{document}
