\documentclass{beamer}
%
% Choose how your presentation looks.
%
% For more themes, color themes and font themes, see:
% http://deic.uab.es/~iblanes/beamer_gallery/index_by_theme.html
%
\mode<presentation>
{
  \usetheme{default}      % or try Darmstadt, Madrid, Warsaw, ...
  \usecolortheme{default} % or try albatross, beaver, crane, ...
  \usefonttheme{default}  % or try serif, structurebold, ...
  \setbeamertemplate{navigation symbols}{}
  \setbeamertemplate{caption}[numbered]
  \setbeamertemplate{footline}[frame number]
  \setbeamertemplate{itemize items}[circle]
  \setbeamertemplate{theorems}[numbered]
  \setbeamercolor*{structure}{bg=white,fg=blue}
  \setbeamerfont{block title}{size=\normalsize}
}

% \newtheorem{proposition}[theorem]{Proposition}
% \theoremstyle{definition}
% \newtheorem{algorithm}[theorem]{Algorithm}
% \newtheorem{idea}[theorem]{Idea}

\usepackage[english]{babel}
\usepackage[utf8]{inputenc}
\usepackage[T1]{fontenc}
\usepackage{aligned-overset}
\usepackage{alltt}
\usepackage{amsmath}
\usepackage{csquotes}
% \usepackage{multicol}
% \usepackage{stmaryrd}
\usepackage{tabularx}

% \renewcommand\tabularxcolumn[1]{m{#1}}
% \newcolumntype{R}{>{\raggedleft\arraybackslash}X}=

\newcommand\logeq{\mathrel{\vcentcolon\Leftrightarrow}}
\def\code#1{\texttt{\frenchspacing#1}}
\def\padding{\vspace{0.5cm}}
\def\spadding{\vspace{0.25cm}}
\def\b{\textcolor{blue}}
\def\r{\textcolor{red}}
\def\g#1{{\usebeamercolor[fg]{block title example}{#1}}}

% fix for \pause in align
\makeatletter
\let\save@measuring@true\measuring@true
\def\measuring@true{%
  \save@measuring@true
  \def\beamer@sortzero##1{\beamer@ifnextcharospec{\beamer@sortzeroread{##1}}{}}%
  \def\beamer@sortzeroread##1<##2>{}%
  \def\beamer@finalnospec{}%
}
\makeatother

\title[Theoretical Computer Science]{Theoretical Computer Science \\ Context-Free Languages}
\author{Jonas Hübotter}
\date{}

\begin{document}

\begin{frame}
  \titlepage
\end{frame}

\begin{frame}{Outline}
 \tableofcontents[subsubsectionstyle=hide]
\end{frame}
% \AtBeginSection[]
%   {
%      \begin{frame}[allowframebreaks]{Plan}
%      \tableofcontents[currentsection, sectionstyle=show/hide, hideothersubsections]
%      \end{frame}
%   }

\section{Overview}

\begin{frame}{Overview}
    \begin{block}{Representations of context-free languages}\pause
        \begin{itemize}
            \item Context-Free Grammar (CFG)\pause
            \item Pushdown Automaton (PDA)
        \end{itemize}
    \end{block}
\end{frame}

\section{Context-Free Grammars}

\subsection{Variables}

\begin{frame}{Variables}
    \begin{definition}
        Given a grammar $G = (V, \Sigma, P, S)$, a variable $X \in V$ is
        \begin{itemize}
            \item \b{generative} if $\exists X \to_G^* w \in \Sigma^*$\pause;
            \item \b{reachable} if $\exists S \to_G^* X$\pause; and
            \item \b{helpful} if it is generative and reachable.
        \end{itemize}
    \end{definition}
\end{frame}

\subsection{Inductive Definition}

\begin{frame}{Inductive Definition}
    Given a context-free grammar $G = (V, \Sigma, P, S)$ with $V = \{A_1, \dots, A_k\}$,\pause\par
    productions $A_i \to w_0 A_{i_1}w_1 \dots w_{n-1} A_{i_n} w_n$\pause\par
    correspond to \begin{align*}
        &u_1 \in L_G(A_{i_1}) \land \dots \land u_n \in L_G(A_{i_n}) \\
        &\implies w_0 u_1 w_1 \dots w_{n-1} u_n w_n \in L_G(A_i).
    \end{align*}\pause
    Hence, $L(G) = L_G(S)$.\pause\par\spadding
    Productions produce words \r{top-down},\par
    inductive definition \textit{produces} words \r{bottom-up}.
\end{frame}

\subsection{Decomposition Lemma}

\begin{frame}{Decomposition Lemma}
    \begin{lemma}[Decomposition Lemma]
        Any derivation of length $n$ of $\beta$ from $\alpha_1 \alpha_2$ may \textit{split} $\beta$ into two separately derivable parts $\beta_1$ and $\beta_2$ at any position.\pause\ Formally:
        \begin{align*}
            \alpha_1 \alpha_2 \to_G^n \beta \iff& \exists \beta_1, \beta_2, n_1, n_2.\ \beta = \beta_1 \beta_2 \land n = n_1 + n_2\ \land \\
                                                & \alpha_1 \to_G^{n_1} \beta_1 \land \alpha_2 \to_G^{n_2} \beta_2.
        \end{align*}
    \end{lemma}
\end{frame}

\subsection{Syntax Tree}

\begin{frame}{Syntax Tree}
    \begin{definition}
        A \b{syntax tree} of a derivation $\to_G$ given $G = (V, \Sigma, P, S)$ is a tree where\pause
        \begin{enumerate}
            \item every leaf is labeled with a symbol in $\Sigma \cup \{\epsilon\}$\pause;
            \item every inner node is labeled with $A \in V$,\par
                assuming its children are $X_1, \dots, X_n \in V \cup \Sigma \cup \{\epsilon\}$, $A \to X_1 \dots X_n \in P$\pause; and
            \item a leaf labeled $\epsilon$ is an only child of its parent.
        \end{enumerate}\pause\spadding
        The \b{border} of a syntax tree is the labels of its leafs concatenated from left to right.
    \end{definition}\pause
    \begin{align*}
        A \to_G^* w &\iff w \in L_G(A) \\
                    &\iff \exists\ \text{syntax tree with root } A \text{ and border } w.
    \end{align*}
\end{frame}

\begin{frame}{Syntax Tree}
    \begin{definition}
        \begin{itemize}
            \item A CFG $G$ is \b{ambiguous} if $\exists w \in L(G)$ that has two distinct syntax trees.
            \item A CFL $L$ is \b{inherently ambiguous} if every CFG $G$ with $L(G) = L$ is ambiguous.
        \end{itemize}
    \end{definition}
\end{frame}

\subsection{Chomsky Normal Form}

\begin{frame}{Chomsky Normal Form}
    \begin{definition}[Chomsky Normal Form]
        All productions are of the form $A \to a$ or $A \to BC$ for $a \in Sigma$ and $A,B,C \in V$.
    \end{definition}\pause

    \begin{block}{Algorithm to convert a CFG to Chomsky Normal Form ($\mathcal{O}(|P|^2)$)}
        \begin{enumerate}
            \item replace every $a \in \Sigma$ occurring in a production with length $>1$ by a non-terminal\pause
            \item replace $A \to B_1 \dots B_k$ (where $k>2$) with $A \to B_1 C_2, C_2 \to B_2, \dots, C_k \to B_k$\pause
            \item remove $\epsilon$-productions (i.e. $A \to \epsilon$)\pause
            \item remove chain productions (i.e. $A \to B$)
        \end{enumerate}
    \end{block}
\end{frame}

\subsection{Other Normal Forms}

\begin{frame}{Other Normal Forms}
    \begin{definition}[Greibach Normal Form]
        All productions are of the form $A \to a A_1 \dots A_n$ for $a \in Sigma$ and $A_1, \dots, A_n \in V$.
    \end{definition}\pause

    \begin{definition}[Backus-Naur Normal Form]
        Allows the use of regular expressions in productions (in addition to symbols).
    \end{definition}
\end{frame}

\subsection{Cocke-Younger-Kasami Algorithm (CYK)}

\begin{frame}{Cocke-Younger-Kasami Algorithm (CYK)}
    Solves the word problem for CFGs.\pause

    \begin{block}{Algorithm ($\mathcal{O}(|w|^3)$)}
        Given $G = (V, \Sigma, P, S)$ in Chomsky normal form and $w = a_1 \dots a_n \in \Sigma^*$.\pause\par
        Define $V_{ij} = \{A \in V \mid A \to_G^* a_i \dots a_j\}$ for $i \leq j$ as the set of all initial symbols that may be used to derive $a_i \dots a_j$.\pause\par
        Then $w \in L_G(A) \iff A \in V_{1n}$.\pause\par\spadding
        Recursive definition of $V_{ij}$:
        \begin{itemize}
            \item base: $V_{ii} = \{A \in V \mid (A \to a_i) \in P\}$\pause
            \item step: \begin{align*}
                V_{ij} = \{A \in V \mid \substack{\exists i \leq k < j, B \in V_{ik}, C \in V_{(k+1)j}.\ \\ (A \to BC) \in P}\}
            \end{align*}
        \end{itemize}
    \end{block}
\end{frame}

\section{Pushdown Automaton (PDA)}

\subsection{Extension Lemma}

\subsection{Decomposition Lemma}

\subsection{CFG $\to$ PDA}

\subsection{PDA $\to$ CFG}

\subsection{Deterministic Pushdown Automaton (DPDA)}

\section{Closure Properties}

\begin{frame}{Closure Properties}
    \begin{theorem}
        Given the context-free languages $L, L_1, L_2$, then the following are also centext-free languages:\pause
        \begin{itemize}
            \item $L_1 L_2$\pause;
            \item $L_1 \cup L_2$\pause; and
            \item $L^*$.
        \end{itemize}
    \end{theorem}\pause

    \begin{theorem}
        Given the deterministic context-free language $L$, then $\bar{L}$ is deterministic context-free.
    \end{theorem}
\end{frame}

\section{Pumping Lemma}

\begin{frame}{Pumping Lemma}
    \begin{lemma}[Pumping Lemma for context-free languages]
        Let $L \subseteq \Sigma^*$ be context-free.\pause\ Then there exists some $n > 0$ such that every $z \in L$ with $|z| \geq n$ can be decomposed into $z = uvwxy$\pause\ such that
        \begin{itemize}
            \item $vx \neq \epsilon$\pause;
            \item $|vwx| \leq n$\pause; and
            \item $\forall i \geq 0.\ uv^iwx^iy \in L$.
        \end{itemize}
    \end{lemma}\pause
    \r{A necessary condition for context-free languages.}
\end{frame}

\begin{frame}{Pumping Lemma}
    \begin{example}[proof structure]
        Assume $L$ is context-free.\par
        Let $n > 0$ be a Pumping Lemma number.\pause\par
        Choose $z \in L$ with $|z| \geq n$.\par
        Define $z = uvwxy$ with $vx \neq \epsilon$ and $|vwx| \leq n$.\pause\par
        Then, $\forall i \geq 0.\ uv^iwx^iy \in L$.\pause\par
        Now, use the last statement to find a contradiction separating all possible cases for $v$ and $x$.
    \end{example}
\end{frame}

\end{document}
